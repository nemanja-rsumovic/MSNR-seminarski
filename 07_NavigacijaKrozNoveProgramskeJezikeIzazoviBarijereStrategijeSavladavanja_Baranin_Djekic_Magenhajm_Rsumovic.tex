% !TEX encoding = UTF-8 Unicode
\documentclass[a4paper]{article}

\usepackage{color}
\usepackage{url}
\usepackage[T2A]{fontenc} % enable Cyrillic fonts
\usepackage[utf8]{inputenc} % make weird characters work
\usepackage{graphicx}
\usepackage{float}
\usepackage{array}
\usepackage{booktabs}
\usepackage{titling}


\usepackage[english,serbian]{babel}
%\usepackage[english,serbianc]{babel} %ukljuciti babel sa ovim opcijama, umesto gornjim, ukoliko se koristi cirilica

\usepackage[unicode]{hyperref}
\hypersetup{colorlinks,citecolor=green,filecolor=green,linkcolor=blue,urlcolor=blue}

\usepackage{listings}

%\newtheorem{primer}{Пример}[section] %ćirilični primer
\newtheorem{primer}{Primer}[section]

\definecolor{mygreen}{rgb}{0,0.6,0}
\definecolor{mygray}{rgb}{0.5,0.5,0.5}
\definecolor{mymauve}{rgb}{0.58,0,0.82}

\lstset{ 
	backgroundcolor=\color{white},   % choose the background color; you must add \usepackage{color} or \usepackage{xcolor}; should come as last argument
	basicstyle=\scriptsize\ttfamily,        % the size of the fonts that are used for the code
	breakatwhitespace=false,         % sets if automatic breaks should only happen at whitespace
	breaklines=true,                 % sets automatic line breaking
	captionpos=b,                    % sets the caption-position to bottom
	commentstyle=\color{mygreen},    % comment style
	deletekeywords={...},            % if you want to delete keywords from the given language
	escapeinside={\%*}{*)},          % if you want to add LaTeX within your code
	extendedchars=true,              % lets you use non-ASCII characters; for 8-bits encodings only, does not work with UTF-8
	firstnumber=1000,                % start line enumeration with line 1000
	frame=single,	                   % adds a frame around the code
	keepspaces=true,                 % keeps spaces in text, useful for keeping indentation of code (possibly needs columns=flexible)
	keywordstyle=\color{blue},       % keyword style
	language=Python,                 % the language of the code
	morekeywords={*,...},            % if you want to add more keywords to the set
	numbers=left,                    % where to put the line-numbers; possible values are (none, left, right)
	numbersep=5pt,                   % how far the line-numbers are from the code
	numberstyle=\tiny\color{mygray}, % the style that is used for the line-numbers
	rulecolor=\color{black},         % if not set, the frame-color may be changed on line-breaks within not-black text (e.g. comments (green here))
	showspaces=false,                % show spaces everywhere adding particular underscores; it overrides 'showstringspaces'
	showstringspaces=false,          % underline spaces within strings only
	showtabs=false,                  % show tabs within strings adding particular underscores
	stepnumber=2,                    % the step between two line-numbers. If it's 1, each line will be numbered
	stringstyle=\color{mymauve},     % string literal style
	tabsize=2,	                   % sets default tabsize to 2 spaces
	title=\lstname                   % show the filename of files included with \lstinputlisting; also try caption instead of title
}

\begin{document}
	
	\title{Navigacija kroz nove programske jezike: izazovi, barijere i strategije savladavanja\\ \small{Seminarski rad u okviru kursa\\Metodologija stručnog i naučnog rada\\ Matematički fakultet}}
	
	\author{
		Tamara Baranin 1029/2025,
		Luka Đekić 1021/2025, \\
		Petar Magenhajm 1126/2025,
		Nemanja Ršumović 1016/2025, \\
		mi251029@alas.matf.bg.ac.rs,
		mi251021@alas.matf.bg.ac.rs, \\
		mi251126@alas.matf.bg.ac.rs,
		mi251016@alas.matf.bg.ac.rs
	}
	
	%\date{9.~april 2015.}
	
	\setlength{\droptitle}{-2cm}
	\maketitle
	
	\abstract{
		Ovaj rad analizira izazove sa kojima se suočavaju kako početnici, tako i iskusni programeri u procesu učenja novih programskih jezika i paradigmi. Istraživanje je sprovedeno putem anketnog upitnika na uzorku od 95 ispitanika, pretežno studenata sa višegodišnjim iskustvom u programiranju, stečenim kroz akademsko obrazovanje, profesionalni rad i samostalno učenje. Fokus rada je na ispitivanju povezanosti izmedju godina iskustva, poznavanja većeg broja programskih jezika i paradigmi, kao i subjektivnih poteškoća pri savladavanju nove programske paradigme. Raznolikost ispitanika u pogledu iskustva, obrazovanja i pristupa učenju omogućava sagledavanje istraživanog problema iz više perspektiva, čime se dodatno povećava validnost donetih zaključaka.
	}
	
	\tableofcontents
	
	\newpage
	
	\section{Uvod}
	\label{sec:uvod}
	
	Učenje programskih jezika predstavlja složen i višeslojan proces koji se može značajno razlikovati od osobe do osobe, u zavisnosti od prethodnog znanja, iskustva i individualnih pristupa učenju. Brzina razumevanja, sposobnost dugoročnog pamćenja, kao i stepen sigurnosti u primeni novog programskog jezika zavise od niza faktora, među kojima se posebno izdvajaju višegodišnje iskustvo u programiranju, poznavanje različitih programskih paradigmi, kao i odabrane strategije učenja. Iako prethodno iskustvo i konceptualno razumevanje paradigmi u velikoj meri olakšavaju savladavanje novih jezika, značajan uticaj ima i način na koji se učenje odvija, bilo kroz kurseve, proučavanje zvanične dokumentacije ili rad na konkretnim projektima \cite{robins2003learning}.
	
	Ipak, izazovi sa kojima se programeri susreću prilikom učenja novih jezika nisu univerzalni, već su u velikoj meri individualni i zavise od više faktora. Upravo iz tog razloga, cilj ovog rada jeste da se, na osnovu većeg broja ispitanika, identifikuju i analiziraju obrasci i korelacije između uočenih prepreka i olakšavajućih faktora u procesu usvajanja različitih programskih jezika i paradigmi. Posebna pažnja posvećena je ispitanicima sa Matematičkog fakulteta, čiji obrazovni program karakteriše snažna teorijska osnova, što potencijalno olakšava apstraktno razumevanje novih koncepata i njihovo povezivanje sa postojećim znanjem. Istovremeno, uključivanje ispitanika sa drugih fakulteta omogućava poređenje različitih obrazovnih pristupa i njihovog uticaja na proces učenja programiranja.
	
	Analiza rezultata ankete ukazuje na to da se kao jedan od ključnih izazova izdvaja učenje programskog jezika koji pripada novoj paradigmi, kao i kvalitet i dostupnost izvora za učenje. Ovi, ali i drugi identifikovani faktori, biće detaljnije ispitani u nastavku rada, sa ciljem da se pruži jasniji uvid u složenost procesa učenja programskih jezika i ponude smernice za efikasnije pristupe učenju u akademskom i samostalnom okruženju.
	
	\section{Pregled osnovnih podataka o ispitanicima}
	
	Većinu ispitanika čine studenti, dok je znatno manji broj onih koji su završili osnovne, master ili doktorske studije. Time se uzorak primarno oslanja na populaciju koja je aktivno uključena u proces učenja i akademskog usavršavanja, što je u skladu sa temom istraživanja. Među studentima postoji raznolikost u pogledu broja položenih ESPB bodova, što ukazuje na prisustvo ispitanika iz različitih faza studiranja, od početnih godina do završnih faza studija. Grafik odnosa broja ESBP poena među ispitanicima se može videti na slici \ref{fig:espb}.
	
	\begin{figure}[H]
		\centering
		\includegraphics[scale=0.8]{espb.png}
		\caption{Položeni ESPB ispitanika koji su studenti}
		\label{fig:espb}
	\end{figure}
	
	Analiza obrazovne strukture ispitanika pokazuje da najveći deo uzorka čine studenti Matematičkog fakulteta (slika \ref{fig:fakulteti}), dok su studenti drugih tehničkih i srodnih fakulteta zastupljeni u manjem broju. Takva raspodela ukazuje da je istraživanje u najvećoj meri obuhvatilo populaciju koja je tokom studija kontinuirano u kontaktu sa programiranjem i srodnim računarskim disciplinama. Ovo doprinosi tome da prikupljeni podaci odražavaju iskustva ispitanika koji se aktivno bave učenjem programskih jezika, što je u skladu sa ciljevima istraživanja i temom seminarskog rada. 
	
	
	\begin{figure}[h!]
		\centering
		\hspace*{-1.5cm}
		\includegraphics[scale=0.5]{fakulteti.png}
		\caption{Raspodele ispitanika po fakultetima}
		\label{fig:fakulteti}
	\end{figure}
	
	
	Kada je u pitanju iskustvo u programiranju, rezultati pokazuju da većina ispitanika ima višegodišnje iskustvo, pri čemu najveći deo navodi da se programiranjem bavi duže od četiri godine. Manji broj ispitanika ima između jedne i četiri godine iskustva, dok je najmanje onih koji se programiranjem bave kraće od jedne godine. Ovakva struktura uzorka omogućava analizu procesa učenja novih programskih jezika iz perspektive ispitanika koji već poseduju značajno prethodno znanje i iskustvo. Grafik odnosa godina isustva među ispitanicima se može uočiti na slici \ref{fig:programiranje}.
	
	
	
	\begin{figure}[h!]
		\centering
		\includegraphics[scale=0.8]{programiranje.png}
		\caption{Godine iskustva}
		\label{fig:programiranje}
	\end{figure}
	
	\section{Izazovi u učenju programskih jezika}
	Učenje programskih jezika predstavlja zahtevan proces koji podrazumeva razvoj sposobnosti analize problema, apstraktnog razmišljanja i razumevanja toka izvršavanja programa. Mnogi učenici, naročito početnici, pristupaju programiranju bez prethodnog planiranja i analize zadatka, fokusirajući se na pojedinačne linije koda umesto na celokupnu strukturu rešenja. Takav pristup često dovodi do nerazumevanja osnovnih programskih koncepata i otežava identifikaciju i ispravljanje grešaka, zbog čega se sintaksni problemi često javljaju kao posledica dubljih konceptualnih poteškoća\cite{lahtinen2005difficulties}.
	\subsection{Uticaj programskih paradigmi na proces učenja}
	Poseban izazov u učenju novih programskih jezika predstavlja savladavanje jezika koji pripadaju drugačijoj programskoj paradigmi. Promena paradigme zahteva prilagođavanje postojećih i usvajanje novih načina razmišljanja o strukturi programa i rešavanju problema. Ovaj proces ne zavisi isključivo od tehničkog znanja ili iskustva, već od sposobnosti da se apstraktni koncepti povežu sa prethodnim znanjem. Zbog toga se problemi u učenju novih paradigmi javljaju i kod iskusnijih programera, a ne samo kod početnika.
	\subsection{Uloga stila učenja i izvora znanja}
	Na uspešnost učenja programskih jezika značajno utiče i stil učenja, kao i kvalitet korišćenih izvora. Pasivni oblici učenja, poput oslanjanja isključivo na teorijska predavanja ili gotove primere, često nisu dovoljni za razvoj dubljeg razumevanja\cite{freeman2014active}.
	Suprotno tome, praktičan rad, rešavanje konkretnih problema i eksperimentisanje sa kodom omogućavaju postepeno građenje sigurnosti i povezivanje teorijskih znanja sa njihovom primenom. Kvalitetna dokumentacija, jasni primeri i dostupnost praktičnih zadataka dodatno olakšavaju proces usvajanja novih programskih koncepata \cite{stackoverflow_survey}.
	\subsection{Završna razmatranja}
	Iako se poteškoće u učenju programskih jezika mogu ispoljavati na različite načine, zajedničko im je to što ne proističu isključivo iz same složenosti jezika, već iz kombinacije kognitivnih zahteva, prethodnog znanja i načina učenja. Razumevanje ovih faktora omogućava sagledavanje učenja programiranja kao postepenog procesa u kome se tehničke veštine razvijaju uporedno sa sposobnošću apstraktnog razmišljanja i rešavanja problema. Upravo iz tog razloga, identifikovanje obrazaca u izazovima i strategijama učenja, na osnovu većeg broja ispitanika, predstavlja važan korak ka unapređenju nastavnih pristupa i efikasnijem usvajanju novih programskih jezika i paradigmi \cite{programming_difficulties}.
	
	
	\section{Analiza ankete}
	U nastavku će biti prikazana analiza sprovedene ankete, identifikovani glavni problemi, kao i preporuke za rešavanje tih problema. Anketu je popunilo 95 ispitanika. Dobijeni podaci omogućavaju detaljniji uvid u navike ispitanika prilikom učenja novih programskih jezika, kao i u njihove subjektivne procene efikasnosti različitih pristupa. Analiza rezultata pruža osnov za razumevanje obrazaca ponašanja i uočavanje faktora koji mogu značajno uticati na uspešnost procesa učenja.
	
	\subsection{Strategije učenja i njihova efikasnost}
	Rezultati ankete ukazuju da ispitanici preferiraju aktivne strategije učenja, odnosno pristupe koji podrazumevaju neposrednu primenu znanja kroz kodiranje i rešavanje problema. Kada posmatramo koje strategije ispitanici najčešće koriste, dominiraju online resursi i samostalno istraživanje: online kursevi (70), zvanična dokumentacija (62), knjige i skripte (56) i čitanje tuđeg koda (54), dok su male samostalne vežbe (42) i rad na konkretnim projektima (41) takođe vrlo zastupljeni. Sve strategije koje su ispitanici naveli prikazane su na slici \ref{fig:strategije}.
	
	\begin{figure}[h!]
		\hspace*{-1cm}
		\includegraphics[scale=0.7]{strategije.png}
		\caption{Strategije koje se koriste prilikom učenja novog jezika}
		\label{fig:strategije}
	\end{figure}
	
	Međutim, kada se ispitanici opredeljuju za najefikasniju strategiju, slika postaje još jasnija: najviše glasova dele online kursevi (21) i rad na konkretnim projektima (21). To sugeriše da pasivno ili informativno učenje (npr. samo čitanje ili gledanje) najčešće dobija smisao tek kada se poveže sa praksom i konkretnim zadatkom \cite{mdnlearning}.
	
	Otvoreni odgovori dodatno potvrđuju ovaj obrazac i naglašavaju vrednost “učenja kroz rad” i iterativnog napredovanja kroz greške:
	\begin{itemize}
		\item \textit{„Izrada konkretnog projekta“}
		\item \textit{„Mali projekti, interaktivno učenje“}
		\item \textit{„Rad na konkretnim primerima i projektima, uz stalno eksperimentisanje i greške iz kojih se uči.“}
		\item \textit{„Bacanje u vatru i rad na konkretnom projektu više pomaže nego suvoparno učenje sintakse.“}
		
	\end{itemize}
	Ovi navodi su u skladu sa nalazom da ispitanici prepoznaju značaj iskustvenog učenja, gde se znanje proverava kroz implementaciju, što vremenom povećava sigurnost i sposobnost samostalnog rešavanja problema.
	
	\begin{figure}[h!]
		\centering
		\includegraphics[scale=0.8]{strategije2.png}
		\caption{Najefikasnije strategije koje se koriste prilikom učenja novog jezika}
		\label{fig:strategije2}
	\end{figure}
	
	
	Iako se u odgovorima jasno vidi sklonost ka učenju kroz praktičan rad i izradu projekata, prilikom procene najefikasnije strategije ne dolazi do jasne dominacije nijednog pristupa. Umesto toga, ispitanici su se opredeljivali za različite metode učenja, pri čemu su online kursevi, samostalne vežbe, rad na projektima i korišćenje literature zastupljeni u sličnom obimu. Ovakav nalaz potvrđuje da efikasnost strategije nije apsolutna kategorija, već zavisi od faze učenja, tipa programskog jezika i individualnog načina usvajanja znanja, zbog čega se različite strategije mogu smatrati podjednako korisnim u različitim okolnostima što je prikazano na slici \ref{fig:strategije2}.
	
	\subsection{Barijere i izazovi u učenju novih programskih jezika}
	Da bi se barijere sagledale preciznije, analizirani su različiti aspekti prilagođavanja novom jeziku (sintaksa, paradigma, tipovi, alati/okruženje, ekosistem, dokumentacija). Kada se za svakog ispitanika identifikuje aspekt koji je dobio najveću ocenu težine, dobija se vrlo informativna slika dominantnih prepreka.
	
	Najčešće isticana „najveća barijera” je:
	\begin{itemize}
		\item Promena programske paradigme
		\item Alati i razvojno okruženje
	\end{itemize}
	Dok su dokumentacija, sintaksa, ekosistem i tipovi navedeni kao najmanja prepreka.
	
	Ovaj nalaz je važan jer sugeriše da poteškoće nisu samo „jezičke” (sintaksa), već da često nastaju u delu koji zahteva promenu načina razmišljanja (paradigma) ili praktično snalaženje u novom okruženju. To objašnjava zašto prelazak na novi jezik ponekad deluje „teže nego što se očekuje”, čak i kada je sintaksa relativno jednostavna.
	
	
	\begin{table}[H]
		\centering
		\caption{Faktori koji predstavljaju prepreku}
		\label{tab:faktori}
		\hspace*{-1.9cm}
		\begin{tabular}{m{6cm}||c|c|c|c|c}
			
			\rule{0pt}{0.5cm} & 1 & 2 & 3 & 4 & 5 \\
			\hline\hline
			\raisebox{0.3cm}{Razumevanje sintakse} \rule{0pt}{1cm}
			& 34(35.8\%) & 38(40\%) & 17(19.9\%) & 5(5.3\%) &  1(1.1\%)  \\
			\hline
			\raisebox{0.3cm}{Promena programske paradigme} \rule{0pt}{1cm} & 14(14.7\%) & 22(23.2\%) & 31(32.6\%) & 26(27.4\%) &  2(2.1\%)  \\
			\hline
			\raisebox{0.3cm}{Razumevanje sistema tipova} \rule{0pt}{1cm} & 31(32.6\%) & 26(27.4\%) & 26(27.4\%) & 11(11.6\%) &  1(1.1\%)  \\
			\hline
			\raisebox{0.3cm}{Alati i razvojno okruženje}  \rule{0pt}{1cm} & 15(15.8\%) & 25(26.3\%) & 29(30.5\%) & 18(18.9\%) &  8(8.4\%)  \\
			\hline
			\raisebox{0.3cm}{Biblioteke i ekosistem} \rule{0pt}{1cm} & 8(8.4\%) & 31(32.6\%) & 44(46.3\%) & 8(8.4\%) &  4(4.2\%)  \\
			\hline
			\raisebox{0.3cm}{Kvalitet ili dostupnost dokumentacije} \rule{0pt}{1cm} & 20(21.1\%) & 23(24.2\%) & 27(28.4\%) & 21(22.1\%) &  4(4.2\%)  \\
			
		\end{tabular}
	\end{table}
	
	
	Na osnovu prikazanih rezultata u tabeli \ref{tab:faktori} dat je pregled faktora koji mogu predstavljati prepreku u učenju novih programskih jezika, pri čemu su ispitanici ocenili stepen težine svakog od navedenih aspekata na skali od 1 do 5 (1 ozna;ava \textit{„nije predstavljalo problem”}, dok 5 označava \textit{„veoma je predstavljalo problem”}).
	
	\subsection{Statistička analiza}
	Na osnovu detaljnije analize prikupljenih podataka mogu se uočiti jasni statistički trendovi. Kada ispitanike grupišemo prema tome da li su upoznati sa konceptom programskih paradigmi i zatim posmatramo njihovu sposobnost prilagođavanja novom programskom jeziku, primećuje se značajna razlika između ove dve grupe.
	
	Rezultati pokazuju da ispitanici koji su upoznati sa konceptom programskih paradigmi u proseku lakše i brže usvajaju novi programski jezik u poređenju sa onima koji nemaju takvo predznanje. Ovakav nalaz ukazuje na to da razumevanje programskih paradigmi predstavlja važan faktor koji olakšava proces učenja i adaptacije na nove tehnologije.

	
	Slično prethodnoj analizi, grupisanjem ispitanika prema dužini bavljenja programiranjem i broju položenih ESPB bodova može se uočiti gotovo linearna zavisnost u odnosu na zahtevnost prilagođavanja novom programskom jeziku. Rezultati pokazuju da duže iskustvo u programiranju, kao i veći broj položenih ESPB bodova, u proseku vodi ka lakšem i bržem upoznavanju sa novim programskim jezicima.
	
	Međutim, zanimljiv izuzetak javlja se kod ispitanika sa najmanje iskustva, koji subjektivno procenjuju da nemaju mnogo poteškoća prilikom učenja novih programskih jezika. Ovaj fenomen može se objasniti na dva načina. Prvo, početnici često nemaju ukorenjene navike, obrasce razmišljanja ili vezanost za određene koncepte i paradigme. Otvorenost ka novim idejama i odsustvo prethodnih ograničenja može olakšati inicijalno upoznavanje sa novim jezikom. Sa druge strane, neiskustvo može dovesti i do pogrešne procene sopstvenog znanja. Ispitanici sa manjim iskustvom mogu steći utisak da dobro poznaju određeni programski jezik, iako je njihovo znanje u suštini površno. Iskusniji programeri, nasuprot tome, češće su samokritični i svesni širine i složenosti oblasti, te imaju strože kriterijume za procenu sopstvenog znanja. Ovo je prikazano na slici \ref{fig:iskustvo}.
	
		\begin{figure}[h!]
		\centering
		\includegraphics[scale=0.5]{zahtevnost_iskustvo.png}
		\caption{Prosečna zahtevnost prilagođavanja u odnosu na iskustvo}
		\label{fig:iskustvo}
	\end{figure}
	
	Analiza prikupljenih podataka pokazuje jasnu i statistički uočljivu zavisnost između učestalosti učenja novih programskih jezika i zahtevnosti prilagođavanja novom jeziku. Ispitanici koji češće uče nove programske jezike u proseku se lakše i brže prilagođavaju novim tehnologijama u poređenju sa onima koji to čine povremeno ili retko.
	
	Ovakav rezultat sugeriše da redovno izlaganje novim programskim jezicima i konceptima razvija određenu vrstu „učilačke rutine“, koja vremenom smanjuje kognitivni napor potreban za usvajanje novih znanja. Drugim rečima, što se pojedinac češće susreće sa procesom učenja novih jezika, to taj proces postaje efikasniji i manje zahtevan.
	
	Rezultati analize strategija učenja pokazuju da se rad na konkretnim projektima izdvaja kao najefikasnija strategija za učenje novog programskog jezika. Ispitanici koji su ovu strategiju naveli kao jednu od korišćenih u proseku su iskazivali manju zahtevnost prilagođavanja novom jeziku u odnosu na ostale strategije.
	
	Pored praktičnog rada, visoko su rangirane i strategije koje podrazumevaju teorijski pristup učenju, pre svega zvanična dokumentacija, kao i knjige i skripte. Ovakav rezultat ukazuje na to da uspešno savladavanje novog programskog jezika zahteva kombinaciju teorijskog razumevanja i praktične primene. Teorija pruža osnovu i strukturu znanja, dok praksa omogućava njegovo učvršćivanje i primenu u realnim problemima.
	

	
	\section{Informacije prikupljene pitanjima otvorenog karaktera}
	Kako bi se dobio dublji uvid u individualna iskustva, stavove i strategije ispitanika koje nije moguće u potpunosti obuhvatiti zatvorenim pitanjima, u okviru ankete prikupljene su i informacije putem pitanja otvorenog karaktera, čija je analiza predstavljena u nastavku.
	\subsection{Razumevanje paradigmi i algoritama je važnije od samog jezika}
	Jedan deo ispitanika smatra da konkretan programski jezik nije presudan, već da je ključno razumevanje algoritama i paradigmi, jer se na toj osnovi relativno lako prelazi na druge jezike.
	Ovakav stav ukazuje na viši nivo apstrakcije u razmišljanju o programiranju, gde se jezici posmatraju kao alati, a ne kao centralni predmet učenja. Ovo je tipično za iskusnije programere, kod kojih se znanje ne vezuje za sintaksu, već za koncepte \cite{algorithms}.
	
	\subsection{Intenzivno pisanje koda kao osnovni mehanizam učenja}
	Veliki broj ispitanika naglašava da je kontinuirano pisanje koda, često ponavljano u različitim formulacijama, najefikasniji način učenja novog jezika.
	Ovo sugeriše da ispitanici doživljavaju učenje programiranja kao proces koji se dominantno odvija kroz praksu i ponavljanje, pri čemu se greške posmatraju kao sastavni deo učenja, a ne kao neuspeh.
	
	
	\subsection{Učenje kroz poznate zadatke u novom jeziku}
	Pojedini ispitanici savetuju da se novi jezik uči kroz rešavanje problema koji su već ranije rađeni u drugim jezicima.
	Ovakav pristup smanjuje opterećenje pri učenju jer je problem već poznat, pa se pažnja može usmeriti na razlike u sintaksi i specifičnosti njegove paradigme.
	
	Neki ispitanici upozoravaju da ne treba mehanički preslikavati rešenja iz drugih jezika, već učiti kako se problemi rešavaju na način koji je prirodan za dati jezik.
	Ovaj komentar ukazuje na svest o tome da svaki jezik ima sopstvenu filozofiju i namenu, te da površno poznavanje sintakse nije dovoljno za efikasno korišćenje jezika u praksi.
	
	\subsection{Ne postoji univerzalna strategija učenja}
	Pojedini odgovori eksplicitno ističu da ne postoji jedinstvena strategija koja odgovara svima i da svako mora pronaći pristup koji mu lično najviše odgovara.
	Ovaj stav potvrđuje da je efikasnost uslovljena individualnim faktorima i prethodnim iskustvom.
	
	\subsection{Strpljenje, vreme i kontinuitet kao ključni faktori}
	U više odgovora naglašava se potreba da se učenju pristupi strpljivo, bez preskakanja osnova, uz redovan i dugoročan rad.
	Ovi komentari ukazuju na shvatanje učenja programiranja kao sporog, ali stabilnog procesa, u kome se dobre navike stečene na početku dugoročno isplate, naročito pri učenju složenijih jezika i paradigmi.
	
	
	\section{Zaključak}
	\label{sec:zakljucak}
	
	Učenje novih programskih jezika predstavlja kontinuiran i višeslojan proces koji se ne završava usvajanjem sintakse, već podrazumeva razvoj načina razmišljanja, prilagođavanje različitim paradigmama i postepeno sticanje sigurnosti u praktičnoj primeni znanja. U savremenom okruženju, u kome se tehnologije brzo menjaju, sposobnost da se efikasno savlada novi programski jezik postaje jednako važna kao i poznavanje samog jezika.
	
	Rezultati ankete jasno ukazuju da ovaj proces  predstavlja složen kognitivni i praktični izazov, ali da se uz odgovarajuće strategije i motivaciju može značajno olakšati. Istraživanje potvrđuje da iskustvo igra ključnu ulogu, ali i da početnici mogu ostvariti napredak ukoliko primene aktivne metode učenja i imaju jasne ciljeve. Najveće poteškoće u učenju novih programskih jezika često se ne vezuju za tehničke detalje, već za promenu ustaljenih obrazaca razmišljanja i prilagođavanje drugačijim konceptima i alatima. 
	
	Takođe, izbor strategije učenja ima značajan uticaj na uspešnost usvajanja novih znanja. Aktivni pristupi, koji uključuju praktičan rad, eksperimentisanje i primenu naučenog na realne ili poznate zadatke, omogućavaju dublje razumevanje i dugoročnije pamćenje gradiva. Istovremeno, važno je naglasiti da ne postoji jedinstvena strategija koja odgovara svima, već da se efikasno učenje oslanja na prilagođavanje individualnim potrebama i iskustvu.
	
	Na kraju, može se zaključiti da uspešno učenje programskih jezika zahteva strpljenje, kontinuitet i spremnost na suočavanje sa greškama kao sastavnim delom procesa. Razumevanje da su koncepti i paradigme važniji od same sintakse ne doprinosi samo efikasnijem učenju pojedinačnih jezika, već i dugoročnom razvoju veština neophodnih u savremenom softverskom okruženju. 
	
	\newpage
	\addcontentsline{toc}{section}{Literatura}
	\appendix
	\bibliography{seminarski} 
	\bibliographystyle{plain}
	%\bibliographystyle{unsrt}

	
	
\end{document}

