

 % !TEX encoding = UTF-8 Unicode

\documentclass[a4paper]{report}

\usepackage[T2A]{fontenc} % enable Cyrillic fonts
\usepackage[utf8x,utf8]{inputenc} % make weird characters work
\usepackage[serbian]{babel}
%\usepackage[english,serbianc]{babel}
\usepackage{amssymb}

\usepackage{color}
\usepackage{url}
\usepackage[unicode]{hyperref}
\hypersetup{colorlinks,citecolor=green,filecolor=green,linkcolor=blue,urlcolor=blue}

\newcommand{\odgovor}[1]{\textcolor{blue}{#1}}

\begin{document}

\title{Navigacija kroz nove programske jezike: izazovi, barijere i strategije savladavanja\\ \small{Tamara Baranin 1029/2025, Luka Đekić 1021/2025, \\ Petar Magenhajm 1126/2025, Nemanja Ršumović 1016/2025}}

\maketitle

\tableofcontents

 
\chapter{Recenzent \odgovor{--- : 4} }
% Krupne primedbe koje su datu su bile dobre, ali kako nismo u mogućnosti da sprovedemo ponovno anketiranje ne možemo da ih uvrstimo u naš rad (nisu praktično primenljive u ovom trenutku).

\section{O čemu rad govori?}
U radu se razmatraju problemi i izazovi u učenju novih programskih jezika i paradigmi. Prikazani su različiti načini i strategije učenja, poput online kurseva, zvanične literature, pisanja i analize koda. Takođe se ističu glavne barijere, kao što su promena programske paradigme i alati razvojnog okruženja, kao i praktični saveti za efikasnije savladavanje novih jezika.

\section{Krupne primedbe i sugestije}
U radu nisu uočene krupne primedbe, jer je rad ispunio sva očekivanja u skladu sa zadatom temom.
\\Kao moguća sugestija za unapređenje rada, moglo bi se osvrnuti na različite programske paradigme, imajući u vidu da one nisu iste složenosti. Neke paradigme su intuitivnije i lakše za savladavanje, pa bi isticanje tih razlika moglo dodatno doprineti boljem razumevanju izazova pri učenju novih programskih jezika.

\odgovor{Slažemo se sa zapažanjem recezenta, ali s obzirom da je anketa već sprovedena i nismo u mogućnosti da dobijemo podatke koji bi nam  detaljnije ukazali na uticaj poznavanja različitih programskih paradigmi na savladavanje novih programskih jezika.}

Takođe, prilikom razmatranja uticaja prethodnog iskustva na proces adaptacije, moglo bi se sagledati u kojoj meri znanje stečeno kroz predavanja, kao i praktična iskustva sa vežbi tokom studiranja, doprinose lakšem i manje izazovnom savladavanju novih programskih jezika i paradigmi. Preciznije, moglo bi se ispitati koliko dobra osnova stečena na fakultetu pomaže u ovom procesu i olakšava učenje.

\odgovor{Odlična primedba. U okviru trećeg dela naše ankete "Strategija učenja" bilo bi dobro da smo dodatno razmatrali uticaj znanja stečenih na fakultetu na proces učenja novih programskih jezika i na taj način dublje analizirali u kojoj meri to znanje olakšava proces adaptacije. Međutim, kako je anketa već sprovedena nismo u mogućnosti da dobijemo te rezultate.}

\section{Sitne primedbe}
Kao sitne primedbe mogu se navesti pojedine tipografske greške. U sažetku rada u reči „između“ umesto slova „đ“ zapisano je „dj“. Zatim, na stranici 3, u poslednjoj rečenici poglavlja 2, zapisano je „isustva“ umesto „iskustva“. Takođe, na dnu strane 5 zapisan je  karakter „;“ umesto slova „č“.

\odgovor{Uočene tipografske greške u radu su ispravljene.}


\section{Provera sadržajnosti i forme seminarskog rada}

\begin{enumerate}
\item Da li rad dobro odgovara na zadatu temu?\\
\newline
Rad odgovara zadatoj temi. U njemu su prikazani problemi koji se javljaju pri učenju novih programskih jezika, kao i različiti načini i strategije učenja. Dodatno kroz pitanja otvorenog karaktera, rad nudi različite ideje za još efikasnije učenje.
 
\item Da li je nešto važno propušteno?\\
\newline
Ne, u radu su obuhvaćene sve najbitnije stavke vezane za navigaciju kroz nove programske jezike.

\item Da li ima suštinskih grešaka i propusta?\\
\newline
Ne, u radu nema suštinskih grešaka ni značajnih propusta, jer su svi ključni delovi rada jasno i pravilno obrađeni.

\item Da li je naslov rada dobro izabran?\\
\newline
Da, naslov rada je dobro izabran. Naslov je informativan, jasan i u potpunosti odgovara temi rada, a upotreba reči „navigacija“ je kreativnija i privlačnija u odnosu na jednostavnije nazive, kao što je „novi programski jezici“. Eventualno, naslov bi se mogao skratiti na „Navigacija kroz programske jezike“.

\item Da li sažetak sadrži prave podatke o radu?\\
\newline
Da, sažetak sadrži prave podatke o radu. U njemu je jasno predstavljena tema rada, kao i struktura ispitanika, uz istaknutu raznolikost njihovog iskustva. Sažetak nas uvodi u povezivanje rezultata ankete sa samom temom rada i motiviše da se rad u celosti pročita.

\item Da li je rad lak-težak za čitanje?\\
\newline
Rad je lak za čitanje, dovoljno formalan, ali istovremeno jasan i razumljiv.

\item Da li je za razumevanje teksta potrebno predznanje i u kolikoj meri?\\
\newline
Da. U suštini, rad je razumljiv i bez posebnog predznanja, jer je jasno napisan. Ipak, za potpunije razumevanje pojedinih pojmova, kao što su programske paradigme, razvojna okruženja i slični termini, poželjno je osnovno predznanje iz oblasti programiranja.

\item Da li je u radu navedena odgovarajuća literatura?\\
\newline
Da, u radu je navedena odgovarajuća i dovoljno obimna literatura, koja obuhvata knjige, naučne članke i online izvore.

\item Da li su u radu reference korektno navedene?\\
\newline
Da, reference su ispravno navedene. U tekstu se pravilno poziva na korišćenu literaturu, kao i na slike grafikona koje se nalaze u drugim delovima teksta.

\item Da li je struktura rada adekvatna?\\
\newline
Da, struktura rada je adekvatna i logično organizovana, sa potrebnim celinama: sažetak, uvod, teorijski okvir, analiza ankete, analiza pitanja otvorenog karaktera, zaključak i literatura.

\item Da li rad sadrži sve elemente propisane uslovom seminarskog rada (slike, tabele, broj strana...)?\\
\newline
Da, rad sadrži sve elemente propisane uslovima seminarskog rada. U radu su prisutne slike, grafikoni i tabele koji doprinose boljem razumevanju rezultata, a broj strana je odgovarajući za obrađenu temu, jer rad nije ni prekratak ni predugačak.


\item Da li su slike i tabele funkcionalne i adekvatne?\\
\newline
Da, slike i tabele su funkcionalne i adekvatne. One su nezavisne od teksta, odnosno moguće ih je samostalno pročitati i razumeti.
\end{enumerate}

\section{Ocenite sebe}
c) srednje upućena
\newline
Smatram da sam srednje upućena u oblast koju recenziram, jer sam se tokom osnovnih studija susretala sa različitim programskim jezicima i programskim paradigmama. Takođe, kao neko ko je na početku svoje radne karijere, i dalje se često susrećem sa učenjem novih tehnologija i jezika, što mi omogućava da razumem temu rada, ali i izazove koji se javljaju u procesu učenja.


\chapter{Recenzent \odgovor{--- ocena: 5}}
% Sama struktura rada je poboljšana i rad je unapređen.

\section{O čemu rad govori?}
% Напишете један кратак пасус у којим ћете својим речима препричати суштину рада (и тиме показати да сте рад пажљиво прочитали и разумели). Обим од 200 до 400 карактера.

Rad istražuje poteškoće pri usvajanju novih programskih jezika, kao i efikasnost različitih strategija učenja, uzimajući u obzir uticaj prethodnog iskustvo. Rad kao ključne izazove ističe usvajanje nove programske paradigme konkretnog jezika i prilagođavanje njegovom okruženju, dok učenje kroz praktičan rad navodi kao najefikasniji pristup u najvećem broju slučajeva.

\section{Krupne primedbe i sugestije}
% Напишете своја запажања и конструктивне идеје шта у раду недостаје и шта би требало да се промени-измени-дода-одузме да би рад био квалитетнији.
Trebalo bi uvrstiti i faktor „kvaliteta ili dostupnosti dokumentacije“ u najznačajnije faktore koji utiču na težinu učenja novog programskog jezika, budući da je njegova važnost ocenjena tek neznatno niže u odnosu na ostale ključne faktore. Detaljnije obrazloženje je dato u sekciji o suštinskim greškama i propustima.

\odgovor{Slažemo se sa primedbom recezenta. U najznačajnije faktore koji utiču na težinu učenja novog programskog jezika uvrstili smo i dati faktor.}

\section{Sitne primedbe}
% Напишете своја запажања на тему штампарских-стилских-језичких грешки
„Ovi navodi su u skladu sa nalazom da ispitanici prepoznaju značaj iskustvenog učenja...”

„Ovakav nalaz potvrđuje da efikasnost strategije nije apsolutna kategorija...” 

„Ovaj nalaz je važan jer sugeriše da poteškoće nisu samo ...” 

Reč nalaz stilski ne odgovara datim kontekstima u srpskom jeziku (bolja alternativa je pronalazak).\newline
\odgovor{Razumemo primedbu recezenta, ali smatramo da navedena reč "nalaz" jeste u datom kotekstu u srpskom jeziku. Eventualno, alternativno možemo reći "rezultat ankete".}

Uočene su sledeće slovne greške:
\begin{itemize}
  \item U opisu rezultata iz tabele došlo je do slovne greške u reči označava (napisano je „ozna;ava”).
  \item „Grafik odnosa godina isustva među ispitanicima se može uočiti na slici 3.” Treba da piše iskustva umesto „isustva”.
\end{itemize}


\colorbox{red}{„Najčešće isticane „najveće barijere” su:}
„Najčešće isticana „najveća barijera” je:
\begin{itemize}
  \item Promena programske paradigme
  \item Alati i razvojno okruženje”
\end{itemize}

Umesto u jednini, prethodna rečenica bi trebalo da bude u množini („su” umesto „je”). \newline


U opisu slike broj 9 došlo je do preklapanja teksta.

\odgovor{Uočene greške u radu su ispravljene.}

\section{Provera sadržajnosti i forme seminarskog rada}
% Oдговорите на следећа питања --- уз сваки одговор дати и образложење

\begin{enumerate}
\item Da li rad dobro odgovara na zadatu temu?\\
Rad korektno odgovara na zadatu temu, budući da obrađuje reprezen-
tativan uzorak (studente) i istražuje konkretne izazove u savladavanju novih pro-
gramskih jezika, analizira najčešće korišćene strategije učenja i povezuje 
ih sa prethodnim iskustvom ispitanika.

\item Da li je nešto važno propušteno?\\
Propušteno je da se eksplicitno odgovori na pitanje „Da li postoje obrasci u tome koji jezici se uče lakše ili teže?”.

\odgovor{Ne slažemo se sa ovom primedbom recezenta, jer rad ne odgovara na temu "Izazovi, barijere, strategije i savladavanja konkretnih programskih jezika".}

\item Da li ima suštinskih grešaka i propusta?\\
Faktor „kvaliteta ili dostupnosti dokumentacije“ nije identifikovan kao je-
dna od „najvećih barijera“ u procesu učenja, iako je tek neznatno manji broj ispitanika (25 u odnosu na 26) njemu dodelio visoke ocene uticaja (4 ili 5) u poređenju sa faktorom „alata i razvojnog okruženja“.

\odgovor{Primedba je usvojena.}

\item Da li je naslov rada dobro izabran?\\
Naslov rada je kvalitetno odabran jer opisuje suštinu istraživanja na precizan i jezgrovit način.

\item Da li sažetak sadrži prave podatke o radu?\\
Sažetak korektno i ispravno sumira rad, jer tačno navodi šta je istraživano, kao i strukturu anketiranog tela.

\item Da li je rad lak-težak za čitanje?\\
Rad je lak za čitanje, jer je sadržaj prezentovan na razumljiv i interesantan način.

\item Da li je za razumevanje teksta potrebno predznanje i u kolikoj meri?\\
Za razumevanje teksta neophodno je barem osnovno poznavanje pojmova kao što su programske paradigme, razvojna okruženja i softverski ekosistemi. Ovo podrazumeva poznavanje elemntarnih informatičkih koncepata.

\item Da li je u radu navedena odgovarajuća literatura?\\
U radu je navedena odgovarajuća literatura.

\item Da li su u radu reference korektno navedene?\\
Reference su korektno navedene. 

\item Da li je struktura rada adekvatna?\\
Ispoštovane su standardne norme strukture seminarskog rada (postojanje sažetka, uvoda, razrade, zaključka itd). 
Nije jasno zašto je poglavlje o izazovima u učenju posebno izdvojeno pre analize ankete, kada se i analiza bavi izazovima u učenju. Ako je cilj bio da ono predstavlja uvod koji u obzir ne uzima rezultate ankete, trebalo bi da bude obrađeno pre poglavlja koje se bavi strukturom anketiranog tela.

\odgovor{Strukturu rada smo izmenili u skladu sa ovom smislenom primedbom.}

\item Da li rad sadrži sve elemente propisane uslovom seminarskog rada (slike, tabele, broj strana...)?\\
Rad ispunjava sve zadate propzicije seminarskog rada, što uklučuje postojanje barem jedne slike, tabele, korektan broj strana...

\item Da li su slike i tabele funkcionalne i adekvatne?\\
Slike i tabele su svrsishodne i kvalitetno ilustruju zaključke i opažanja autora.

\end{enumerate}

\section{Ocenite sebe}
% Napišite koliko ste upućeni u oblast koju recenzirate: 
% a) ekspert u datoj oblasti
% b) veoma upućeni u oblast
% c) srednje upućeni
% d) malo upućeni 
% e) skoro neupućeni
% f) potpuno neupućeni
% Obrazložite svoju odluku
U istaživanu oblast sam srednje upućen, budući da sam kroz praksu više puta učio nove programske jezike, ali nisam proučavao literaturu o efikasnosti metoda učenja. Moji stavovi proizilaze iz ličnog iskustva, ali nisu naučno utemeljeni.

\chapter{Recenzent \odgovor{--- ocena: \colorbox{red}{ }} }


\section{O čemu rad govori?}
% Напишете један кратак пасус у којим ћете својим речима препричати суштину рада (и тиме показати да сте рад пажљиво прочитали и разумели). Обим од 200 до 400 карактера.
    Kroz čitanje ovog rada saznajemo koji faktori utiču na težinu usvajanja novih programskih jezika i koje strategije učenja su ispitanici naveli kao najkvalitetnije. Autori daju sliku o tome koliko predhodno iskustvo sa određenom paradigmom, učestalost učenja novih jezika i godine programiranja utiču na lakoću usvajanja novih jezika. Ispituju efikasnost različitih strategija učenja na osnovu odgovora ispitanika i primećuju da je praktični rad na projektima dominantna metoda učenja. Autori daju i kratku analizu otvorenih odgovora ispitanika kako bi prikazali šarenolikost iskustava koja studenti programiranja imaju kada je učenje programskih jezika u pitanju. 

\section{Krupne primedbe i sugestije}
% Напишете своја запажања и конструктивне идеје шта у раду недостаје и шта би требало да се промени-измени-дода-одузме да би рад био квалитетнији.
\colorbox{red}{POGLAVLJE 3 JE SADASNJE POGLAVLJE 2...... NAKNADNO OVO ODRADITI}
\begin{enumerate}
    \item Poglavlje 3.1 iznosi činjenice u vezi povezanosti promene paradigme sa težinom usvajanja novog jezika ali se nigde ne navodi da li je to zaključak autora ovog rada nastao analizom odgovora na anketu ili je to znanje koje su usvojili čitanjem literature. 
    
    \odgovor{ILI LITERATURA ILI NASA ZAPAZANJA.......................]}
    
    \item Mislim da je poglavlje 3 moglo više da se poveže sa podacima koji su prikupljeni anketom pošto je ona sadržala zanimljiva pitanja vezana baš za prepreke u usvajanju novog programskog jezika ili da ovaj deo objedini sa delovima poglavlja 4 gde se ti podaci analiziraju.
    
    \odgovor{poglavlje 4 je analiza ankete i zaključci - poglavlje 3 kao uvod tome}
    \item Volela bih da vidim kako se saznanja do kojih su autori došli svojim istraživanjem slaže ili odudara od onoga što su saznali kroz članke koje su naveli u literaturi. 
    
\end{enumerate}

\section{Sitne primedbe}
% Напишете своја запажања на тему штампарских-стилских-језичких грешки
\begin{enumerate}
    \item U opisu tabele 1 je slovo `č` zamenjeno karakterom `;`.
    
    \odgovor{Uočena greška je ispravljena}
    
    \item Na slici 5 se ne vide svi procenti.
    
    
    \odgovor{Uočena greška je ispravljena}
    
    \item Labele na slici 9 su preklopljene na par karaktera, pa je čitanje otežano.
    
    \odgovor{Uočena greška je ispravljena}
    
    \item Link za referencu broj 6 ne vodi do prave stranice na kojoj se rad nalazi.
    
    \odgovor{Postavljeni link je u trenutku pisanja rada bio ispravan, ali je ResearchGate naknadno izvršio preusmeravanje, jer je promenio lokaciju datog rada. Postavljen je novi link koji sada vodi do stranice na kojoj se rad sada nalazi.}
  
    
\end{enumerate}

\section{Provera sadržajnosti i forme seminarskog rada}
% Oдговорите на следећа питања --- уз сваки одговор дати и образложење

\begin{enumerate}
\item Da li rad dobro odgovara na zadatu temu?\\
    Da, mislim da rad jasno analizira okolnosti koje otežavaju usvajanje novih programskih jezika, kao i one koje idu u prilog lakšem učenju. 
\item Da li je nešto važno propušteno?\\
    Ne.
\item Da li ima suštinskih grešaka i propusta?\\
    Ne, mislim da nema propusta i grešaka.
\item Da li je naslov rada dobro izabran?\\
    Da, mislim da je naslov adekvatan za tekst koji ga prati.
\item Da li sažetak sadrži prave podatke o radu?\\
    Da, sažetak daje tačne informacije o radu ali mislim da treba pomenuti i strategije učenja novih programskih jezika pošto je dobar deo rada posvećen njima. 
\item Da li je rad lak-težak za čitanje?\\ 
      Rad je lak za čitanje jer nema nepoznatih pojmova koji se moraju poznavati detaljno, tok misli je jasan i rečenice su smislene. 
\item Da li je za razumevanje teksta potrebno predznanje i u kolikoj meri?\\
    Za razumevanje teksta nije potrebno opsežno prethodno znanje, može se čitati i bez stručnog znanja vezanog za programiranje. Postoje pojmovi koji su vezani za struku, ali se rad može razumeti i bez dubokog poznavanja pojmova. 
\item Da li je u radu navedena odgovarajuća literatura?\\
    Literatura odgovara temi koja se obrađuje jer se radovi bave baš poteškoćama u usvajanju programerskih veština ili nude načine da se prevaziđu te prepreke.  
\item Da li su u radu reference korektno navedene?\\
    Sve sem linka u referenci broj 6 je navedeno ispravno, način citiranja odgovara tipu literature.
    
        \odgovor{Postavljeni link je u trenutku pisanja rada bio ispravan, ali je ResearchGate naknadno izvršio preusmeravanje, jer je promenio lokaciju datog rada. Postavljen je novi link koji sada vodi do stranice na kojoj se rad sada nalazi.} 
\item Da li je struktura rada adekvatna?\\
    Rad zadovoljava formu koja je zadata, naslovna strana sadrži sve sto je neophodno, postoji uvod, zaključak i razrada koja je podeljena u nekoliko poglavlja, kao i lista referenci. Ipak mi se čini da je količina segmenata nepotrebno velika; poglavlja 3 i 5 su nepotrebno rascepkana na podpoglavlja što daje nadu da će tema iz naslova biti detaljno analizirana ali podnaslov bude praćen samo jednim pasusom.
    
    \odgovor{\colorbox{red}{ODRADITI ZAJEDNO}}
\item Da li rad sadrži sve elemente propisane uslovom seminarskog rada (slike, tabele, broj strana...)?\\
    Da, rad sadrži broj stranica koji je u dozvoljenom opsegu, kao i potreban broj slika, tabela i referenci, kao i potrebne tipove literature. Postoje slike i tabele koje su originalne. 
\item Da li su slike i tabele funkcionalne i adekvatne?\\
     Jesu jer se na njih referiše u tekstu i služe za bolje objašnjenje ideje koja se razrađuje u poglavlju. Kvalitet nekih slika je niži, pa se prilikom uvećavanja gubi oštrina, ali to ne utiče na čitljivost ili razumevanje, već je samo estetsko opažanje.
\end{enumerate}

\section{Ocenite sebe}
% Napišite koliko ste upućeni u oblast koju recenzirate: 
% a) ekspert u datoj oblasti
% b) veoma upućeni u oblast
% c) srednje upućeni
% d) malo upućeni 
% e) skoro neupućeni
% f) potpuno neupućeni
% Obrazložite svoju odluku
Smatram sebe srednje upućenim recenzentom jer sam tokom studija imala priliku da učim jezike iz različitih paradigmi i dosta različitih sintaksi, pa mi iskustvo daje znanje o preprekama i otežavajućim okolnostima koje sam lično iskusila. To iskustvo nije dovoljno da bih se svrstala u poznavaoce ove teme jer, kao što su i autori rada primetili, učenje je jako individualna stvar i manjka mi uopštenije znanje. Takođe nisam upoznata sa literaturom vezanom za ovu temu, van one koja je navedena u ovom radu.   


\end{document}
